\documentclass[12pt]{article}
\usepackage[utf8]{inputenc}
\usepackage{setspace}
\usepackage{indentfirst}

\usepackage[T1]{fontenc}
\title{Documentação Atividade 1}
\author{Gabriel Carneiro}
\date{Outubro 2020}
\begin{document}
\setlength{\footskip}{100pt}

\maketitle 

\section{Introdução}
\begin{spacing}{1.5}
	O trabalho consiste em uma análise de um grafo de relaçoes num grupo de golfinhos.

\section{Como utilizar}
	Basta digitar num terminal:
\begin{verbatim}
$ python3 grafo.py
\end{verbatim}

\subsection{Entrada}
	A entrada do programa consiste num arquivo de GML (dolphins.gml) que é lido automaticamente pelo programa.

\end{spacing}

\section{Resultados}
	Os resultados na análise são os seguintes:
\begin{itemize}
	\item 
		Excentricidade:Beak: 6, Beescratch, Bumper, CCL, Cross, DN16, DN21, DN63, Double, Feather, Fish, Five, Fork, Gallatin, Grin, Haecksel, Hook, Jet, Jonah, Knit, Kringel, MN105, MN23, MN60, MN83, Mus, Notch, Number1, Oscar, Patchback, PL, Quasi, Ripplefluke, Scabs, Shmuddel, SMN5, SN100, SN4, SN63, SN89, SN9, SN90, SN96, Stripes, Thumper, Topless, TR120, TR77, TR82, TR88, TR99, Trigger, TSN103, TSN83, Upbang, Vau, Wave, Web, Whitetip, Zap, Zig, Zipfel
	\item Raio: 5
	\item Diâmetro: 8
	\item Centroide: SN100
	\item Centro: Beescratch, DN63, Knit, Number1, Oscar, PL, SN100, SN89, SN9, Upbang
	\item Periferia: Cross, Five, SMN5, TR120, TR88, TSN83, Whitetip, Zig
\end{itemize}

\end{document}
