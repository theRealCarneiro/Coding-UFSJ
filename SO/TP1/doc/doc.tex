\documentclass[a4paper, 12pt]{article}

\usepackage[portuges]{babel}
\usepackage[utf8]{inputenc}
\usepackage{amsmath}
\usepackage{indentfirst}
\usepackage{graphicx}
\usepackage{multicol,lipsum}
\newcommand{\SubItem}[1]{
    {\setlength\itemindent{15pt} \item[-] #1}
}
\begin{document}
%\maketitle
\begin{titlepage}
	\begin{center}
			\Huge{Universidade Federal de São João del Rei}\\
		\large{Campus Tancredo de Almeida Neves}\\ 
		\large{Departamento Ciência da Computação}\\ 
		\vspace{15pt}
        \vspace{95pt}
        \textbf{\LARGE{Trabalho Sistema Operacional\\Gerenciamento de processos}}\\
		%\title{{\large{Título}}}
		\vspace{3,5cm}
	\end{center}
	
	\begin{flushleft}
		\begin{tabbing}
			Aluno: Gabriel Carneiro \\Aluno: Felipe Samuel\\ Aluno: Andre Luiz \\
			Professor: Rafael Sachetto\\
	\end{tabbing}
 \end{flushleft}
	\vspace{1cm}
	
	\begin{center}
		\vspace{\fill}
			 30 Outubro\\
		 2020
			\end{center}
\end{titlepage}
%%%%%%%%%%%%%%%%%%%%%%%%%%%%%%%%%%%%%%%%%%%%%%%%%%%%%%%%%%%
\newpage
% % % % % % % % % % % % % % % % % % % % % % % % % %
\newpage
\tableofcontents
\thispagestyle{empty}

\newpage
\pagenumbering{arabic}
% % % % % % % % % % % % % % % % % % % % % % % % % % %
\section{Introdução}
O problema proposto foi desenvolver um simulador de gerenciamento de processos. Para atender todas as requisições propostas, implementamos o trabalho da seguinte maneira:
\begin{itemize}
        \item 2 tipos de processos principais: commander, processo manager.
        \item 2 tipos de processos "secundários": o processo simulado e o processo reporter. 
        \item 5 funções principais de gerenciamento: Criação de processos, substituição do processo atual por outro processo, transição de estados de um processo, escalonamento e troca de contexto.
        \item 4 chamadas de sistema do Linux: fork(), wait(), pipe(), sleep().
        

    \end{itemize}
\subsection{Como compilar e executar o programa}
Para compilar e executar o programa basta usar estes comandos num terminal
\begin{verbatim}
Para ir até o diretório do trabalho
$ cd caminho/para/a/pasta

Para compilar
$ make

Para usar o teclado como entrada principal:
$ ./process_commander

Para usar um arquivo de texto:
$ ./process_commander < arquivo.txt
\end{verbatim}
\newpage

\section{Funcionalidades}
O programa consiste em dois elementos principais: o processo commander e o processo manager, a execução do simulador é iniciada pelo processo comandar, e o controle dos processos simulados é feito pelo processo manager. \\
Dito isso faremos uma breve análise das funções implementadas a seguir, posteriormente aprofundando um pouco mais em sua funcionalidade.

\subsection{Processos}

 \subsubsection{Commander}  O processo commander controla como será o andamento do processo manager e do processo simulado, ele primeiramente cria um pipe e um processo do tipo manager, após isso ele lê 1 comando por segundo da entrada principal e envia para o processo manager através do pipe, os comandos são os seguintes:
 \begin{itemize}
 \item \textbf{Q}: Encerra uma unidade de tempo e aciona a próxima instrução do processo simulado
 \item \textbf{U}: Desbloqueia o primeiro processo na fila de processos bloqueados
 \item \textbf{P}: Cria um processo do tipo reporter que imprime o estado atual do sistema
 \item \textbf{T}: Imprime o tempo de retorno médio, cria um processo do tipo reporter e finaliza o simulador
 \end{itemize}
 São usadas 4 chamadas de sistema no processo commander:
 \begin{itemize}
 \item \textbf{fork():} Cria uma cópia do processo em execução:

 \item \textbf{wait():} Espera o processo filho terminar a execução

 \item \textbf{pipe():} Envia uma variável para o processo filho ou algum outro processo.

 \item \textbf{sleep():} Faz o processo ficar "parado", por uma quantidade n de segundos
 \end{itemize}
\subsubsection{Manager} Simula 5 funções de gerenciamento de processos:
\begin{itemize}
\item \textbf{Criação de um processo:} Lê um arquivo com o programa e guarda as instruções, tamanho do arquivo e nome do arquivo na struct CPU e insere-a na tabela PCB (que é o array list pbc\_list).

\item \textbf{Substituição do processo atual por outro processo(troca de imagem):} Substitui o código do programa atual pelo código do programa especificado 

\item \textbf{Transição de estados de um processo:} Um processo sempre está em um dos três possíveis estados:
\begin{itemize}
	\item \textbf{Executando(E):} É a condição do processo que está atualmente na CPU.
	\item \textbf{Pronto(P):} É a condição no qual um processo não esta em execução mas pode ser escalonado a qualquer momento.
	\item \textbf{Bloqueado(B):} É o estado de um processo que estava em execução e executou a instrução B, fazendo com que sua execução seja interrompida e seu estado seja salvo na memoria e o processo seja enviado para a fila de bloqueados.
\end{itemize}

\item \textbf{Escalonamento:}
O escalonador escolhe quem deve executar a partir da lista de prontos com uma fatia de tempo a cada 2 segundos por processo.

\item \textbf{Troca de contexto:} A troca de contexto ocorre quando o processo simulado usa a instrução F (seção 2.1.3 item 6), ela cria uma cópia do processo em execução (pai) e insere na tabela PCB, e começa a executar esse processo filho que é a exata cópia do processo pai.
\end{itemize}


\subsubsection{Simulado} Representa cada processo simulado em um programa, que opera o valor de uma única variável inteira, ou seja o estado de uma variável em um determinado instante de tempo T equivale ao valor da sua variável inteira e o valor de seu contador de programa. Neste caso o processo é armazenado uma sequencia de sete instruções:
\begin{enumerate}
\item \textbf{S} n: Atualiza o valor da variável inteira para n.
\item \textbf{A} n: Soma n na variável inteira
\item \textbf{D} n: Subtrai n Na variável inteira
\item \textbf{B}: Bloqueia o processo simulado
\item \textbf{E}: Termina o processo simulado
\item \textbf{F} n: Cria um novo processo simulado, sendo este a copia exata do processo pai.
\item \textbf{R} arquivo: Substitui o processo atual pelo programa no arquivo com nome arquivo.
\end{enumerate}

\subsubsection{Reporter:} Imprime o estado em que o sistema se encontra.
\begin{footnotesize}
\begin{verbatim}
*****************************************************************************
*****************************************************************************
TEMPO ATUAL: 
PROCESSO EXECUTANDO:
pid     ppid    valor   tempo inicio    CPU usada ate agora     nome processo

PROCESSO BLOQUEADOS:
PROCESSO PRONTOS:
*****************************************************************************
\end{verbatim}
\end{footnotesize}
\newpage
\section{Estruturas e funções}
\subsection{Estruturas de dados}
\subsubsection{Programa}
\begin{verbatim}
typedef struct {
    char comando;
    char *valor;
} programa;
\end{verbatim}
    O código do programa fica armazenado na struct programa, a variável char comando guarda o comando e a variável char *valor guarda os argumentos do comando.

    		
\subsubsection{CPU}
 \begin{verbatim}
typedef struct{
    programa *processo;
    char *nome_arquivo;
    int tamanho;
    int chave;
    int contador;
    int tempo_total;
    int tempo_atual;
} CPU;
\end{verbatim}
    Essa estrutura armazena as informações cruciais para o funcionamento correto do processo, e a execução do processo simulado fica confinada à ela. O código do programa fica armazenado na struct programa *processo, o nome do arquivo na variável char *nome\_arquivo, o número de instruções do programa fica na variável tamanho, o valor inteiro que representa o estado atual num instante de tempo é a variável chave, a variável int contador guarda qual a instrução atual do programa, o int tempo\_total armazena a quantidade de tempo total que o processo ficou na CPU e a variável inteiro tempo\_atual armazena o tempo que o programa está executando na fatia de tempo atual.
\newpage
\subsubsection{Tabela PCB}
\begin{verbatim}
typedef struct{
    CPU *programa;
    int pid;
    int ppid;
    int estado;
    int *contador;
    int tempo_inicio;
} tabela_pcb;
\end{verbatim}
    Essa estrutura armazena todas as informações do processo, inclusive a estrutura CPU que contém o programa. A tabela\_pcb armazena uma struct CPU que aponta para onde está armazenado um processo, o inteiro pid que é o id do processo, o inteiro ppid que é o id do processo pai, um inteiro estado que representa o estado atual do precesso, um inteiro contador que aponta para o contador de programa do processo e o inteiro que armazena o tempo de inicio do processo.
\subsubsection{Array list}
\begin{verbatim}
typedef struct {
    tabela_pcb *pcb;
    int tamanho;
    tabela_pcb *pcb_atual;
} array_list;
\end{verbatim}
    Essa estrutura armazena um lista de arranjo de todos os programas que estão nos 3 possíveis estados de execução (executando, pronto e bloqueado), foi usado um array por ser a melhor opção quando se tem que fazer várias consultas por causa princípio de localidade.
\subsubsection{Fila}
\begin{verbatim}
typedef struct {
     CPU *chave;
     struct fila *prox;
} fila;
\end{verbatim}
    Essa estrutura armazena uma fila de programas, ela é usada no programa para armazenar a fila de processos prontos e de processos bloqueados.

\subsection{Variáveis Globais}
 
     \subsubsection{tempo} É um inteiro inicializado com 0 que armazena o tempo de execução do simulador.
     \subsubsection{executando} É uma struct CPU que armazena o processo que está atualmente em execução.
     \subsubsection{estado\_pronto}  É uma fila que armazena os processos que estão no estado pronto.
     \subsubsection{estado\_bloqueado} É uma fila que armazena os processos que estão no estado bloqueado.
     \subsubsection{pcb\_list} É uma lista de arranjos que armazena a tabela PCB.
     \subsubsection{multiplicador\_pcb} É um inteiro que serve para ser multiplicado pelo tamanho da tabela PCB quando ela estiver cheia.
 

\subsection{Funções}
 \subsubsection{void printa\_processo()}
    Printa informações básicas do processo. A complexidade desta função é O(1).
    
 \subsubsection{void insere\_pcb(CPU *c)}
    Insere o processo colocado como parâmetro da função na tabela PCB, caso ela esteja vazia ele é inserido na posição 0, caso ela n esteja vazia mas também não esteja cheia, o processo é inserido na primeira posição vaga do vetor, e no caso da tabela PCB estar cheia, ela realoca seu tamanho usando a seguinte formula: tamanho\_atual * multiplicado\_pcb (que aumenta em 1 toda vez que a tabela é realocada). A complexidade desta função em seu pior caso é O(1).
    
 \subsubsection{void remove\_pcb(CPU *c)}
    Procura pelo enderenço do processo usado colocado como parâmetro da função na tabela PCB, e caso seja encontrado, ele é removido, a memoria alocada é liberada e os elementos seguintes na tabela são movidos para a posição correta. A complexidade desta função em seu pior caso é O(n).
    
 \subsubsection{tabela\_pcb *busca\_pcb(CPU *c)}
    Procura pelo endereço do processo colocado como parâmetro da função na tabela PCB, e caso seja encontrado, é retornado o elemento, caso não seja encontrado, é retornado o programa que está atualmente em execução. A complexidade desta função em seu pior caso é O(n).
    
 \subsubsection{int line\_count(char *fileName)}
    Conta o número de linhas no arquivo, para ser armazenado na variável tamanho na struct CPU. A complexidade desta função em seu pior caso é O(n)
    
 \subsubsection{void enfilera(fila **f, CPU *chave)}
    Insere um processo em uma fila. A complexidade desta função em seu pior caso é O(n).
    
 \subsubsection{fila *desenfilera(fila **f)}
    Remove o primeiro elemento de uma fila e retorna ele. A complexidade desta função em seu pior caso é O(1).
    
 \subsubsection{void printa\_fila(fila *f)}
    Imprime os elementos que estão em uma fila. A complexidade desta função em seu pior caso é O(n).
    
 \subsubsection{void printa\_comando(programa p)}
    Imprime uma determinada instrução de um programa. A complexidade desta função em seu pior caso é O(1).
    
 \subsubsection{CPU cria\_processo(char *prog)}
    Lê o arquivo com o nome especificado como argumento da função, e armazena informações numa struct CPU, mais informações sobre o que é guardado nessa estrutura estão na seção 3.1.2. A complexidade desta função em seu pior caso é O($n^2$), sendo n o número de linhas no arquivo e m o tamanho do argumento de cada instrução
    
 \subsubsection{void bloqueia\_processo()}
    Bloqueia o processo que está atualmente em execução e insere ele na fila de bloqueados. A complexidade desta função em seu pior caso é O(n).
    
 \subsubsection{void desbloqueia\_processo()}
    Remove o primeiro processo na fila de bloqueados e insere na fila de processos prontos. A complexidade desta função em seu pior caso é O(1).
    
 \subsubsection{void troca\_de\_imagem(CPU *c)}
    Substitui o programa em execução pelo processo especificado como parâmetro da função. A complexidade desta função em seu pior caso é O(n).
 
 \subsubsection{void troca\_de\_contexto()}
    Faz uma cópia do processo em execução e insere na tabela PCB, essa nova cópia é então executada e o processo pai é inserido na fila de processos prontos. A complexidade desta função em seu pior caso é O(n).
    
 \subsubsection{void escalona\_processo()}
    Caso o processo acabe sua execução ou exceda o limite de tempo de sua fatia, ele é inserido na fila de processos prontos e o primeiro processo na fila de processos prontos é executado. A complexidade desta função em seu pior caso é O(n).
    
 \subsubsection{void encerra\_processo()}
    Encerra a execução do processo atual, desaloca a memoria, remove o processo da tabela PCB e chama a função de escalonamento para que um novo processo seja executado. A complexidade desta função em seu pior caso é O(n).
    
 \subsubsection{void reporter()}
    Imprime o estado atual do sistema, como descrito na seção 2.1.4. A complexidade desta função em seu pior caso é O(n).
    
 \subsubsection{void substitui\_processo()}
    Substitui o código do processo atual por outro código, a diferença dessa função para a troca de imagem é que essa função troca o código do programa até mesmo na tabela PCB, enquanto a troca de imagem apenas troca o processo que está em execução. A complexidade desta função em seu pior caso é O($n^2$), pois ela chama a função cria processo.
    
 \subsubsection{void executa\_processo()}
    Essa é a função responsável pela execução de um processo simulado, ela que interpreta os comando e chama as respectivas funções. A complexidade desta função em seu pior caso é O($n^2$), pois ela pode executar todas as funções do programa.
    
 \subsubsection{int main()}
    Essa é a função principal, no main é que é feito a leitura do comando enviado no pipe pelo processo comander e é nele que este comando é interpretado. O main chama a função executa\_processo em resposta à instrução Q, chama a função desbloqueia\_processo em resposta à instrução U, cria um processo do tipo reporter em resposta à instrução P e em resposta à instrução T ele cria um processo reporter e finaliza o simulador após o fim deste processo. 
\newpage
\begin{footnotesize}
\section{Analise de dados}
A seguir, os dados avaliados de acordo com o tempo de execução (lembrando que cada processo tem uma fatia de tempo de 2 segundos):
\begin{verbatim}
Arquivo init
S 1000
A 23
F 1
R programa_a
A 21
D 20
E

Arquivo programa_a
S 100
A 19
A 20
B
A 21
E


Q 
P
*****************************************************************************
Estado do sistema:
*****************************************************************************
TEMPO ATUAL: 1
PROCESSO EXECUTANDO:
pid     ppid    valor   tempo inicio    CPU usada ate agora     nome processo
0       0       1000    0               1                       init
PROCESSO BLOQUEADOS:
PROCESSO PRONTOS:
*****************************************************************************

Q  
Q - Nesse momento o algoritmo troca de contexto em decorrência da instrução F
P
*****************************************************************************
Estado do sistema:
*****************************************************************************
TEMPO ATUAL: 3
PROCESSO EXECUTANDO:
pid     ppid    valor   tempo inicio    CPU usada ate agora     nome processo
1       0       1023    3               3                       init
PROCESSO BLOQUEADOS:
PROCESSO PRONTOS:
0       0       1023    0               3                       init
*****************************************************************************

Q - Substitui o código do programa init (có                                                             pia do init original) pelo código do programa_a
P
*****************************************************************************
Estado do sistema:
*****************************************************************************
TEMPO ATUAL: 4
PROCESSO EXECUTANDO:
pid     ppid    valor   tempo inicio    CPU usada ate agora     nome processo
1       0       0       3               0                       programa_a
PROCESSO BLOQUEADOS:
PROCESSO PRONTOS:
0       0       1023    0               3                       init
*****************************************************************************

Q  
Q
Q
Q
Q
Q - Neste momento o processo do programa_a é bloqueado
P
*****************************************************************************
Estado do sistema:
*****************************************************************************
TEMPO ATUAL: 10
PROCESSO EXECUTANDO:
pid     ppid    valor   tempo inicio    CPU usada ate agora     nome processo
0       0       1024    0               5                       init
PROCESSO BLOQUEADOS:
1       0       139     3               4                       programa_a
PROCESSO PRONTOS:
*****************************************************************************

Q - Termina a execucão do programa init
P
*****************************************************************************
Estado do sistema:
*****************************************************************************
TEMPO ATUAL: 11
PROCESSO EXECUTANDO:
pid     ppid    valor   tempo inicio    CPU usada ate agora     nome processo
PROCESSO BLOQUEADOS:
1       0       139     3               4                       programa_a
PROCESSO PRONTOS:
*****************************************************************************



U - É desbloqueado o primeiro processo na fila de bloqueados
P
*****************************************************************************
Estado do sistema:
*****************************************************************************
TEMPO ATUAL: 11
PROCESSO EXECUTANDO:
pid     ppid    valor   tempo inicio    CPU usada ate agora     nome processo
PROCESSO BLOQUEADOS:
PROCESSO PRONTOS:
1       0       139     3               4                       programa_a
*****************************************************************************

Q - O programa_a sai da fila de prontos e volta a ser executado
P
*****************************************************************************
Estado do sistema:
*****************************************************************************
TEMPO ATUAL: 12
PROCESSO EXECUTANDO:
pid     ppid    valor   tempo inicio    CPU usada ate agora     nome processo
1       0       160     3               5                       programa_a
PROCESSO BLOQUEADOS:
PROCESSO PRONTOS:
*****************************************************************************

Q - O programa_a termina sua execução 
P
*****************************************************************************
Estado do sistema:
*****************************************************************************
TEMPO ATUAL: 13
PROCESSO EXECUTANDO:
pid     ppid    valor   tempo inicio    CPU usada ate agora     nome processo
PROCESSO BLOQUEADOS:
PROCESSO PRONTOS:
*****************************************************************************

T - Fim do simulador de processos
*****************************************************************************
Estado do sistema:
*****************************************************************************
TEMPO ATUAL: 13
PROCESSO EXECUTANDO:
pid     ppid    valor   tempo inicio    CPU usada ate agora     nome processo
PROCESSO BLOQUEADOS:
PROCESSO PRONTOS:
*****************************************************************************
\end{verbatim}
\end{footnotesize}
\newpage
\section{Conclusão}
O objetivo deste trabalho foi construir um algoritmo de gerenciamento de recursos, porém houve etapas de difícil execução, principalmente na troca de contexto onde não é tão trivial a abstração e implementação na linguagem C, no decorrer do desenvolvimento conseguimos atingir o objetivo e fazer análise dos dados obtidos através de testes previamente estipulados. Portanto, este trabalho abre uma uma oportunidade de simular e compreender os processos executados no Linux.
\newpage
\section{Bibliografia}
\footnotesize{
Todo material utilizado foi obtido através do manual do LINUX, nenhum outro material foi consultado para o desenvolvimento do trabalho.

}
\end{document}
